\chapter{Introduction}\label{ch:intro}

Reinforcement Learning is starting to be prevalent in many fields and applications, including Autonomous Driving, Robotics, Transportation, in Science and Engineering, and many more. A lot of recent research has started in applying Reinforcement Learning in Space and space fairing vehicles. The benefits are better computational requirements from switching from classical numerical methods to a Reinforcement Learning based approach. The goal is to create an environment for detailed evaluations of RL algorithms, current, and future, in Station-Keeping, maintaining a satellite's orbit.

With SpaceX's Starlink, their satellite internet project, we are putting an increasing number of satellites in Low Earth Orbit. In the case of Starlink, this allows their satellites to deliver low latency high bandwidth internet compared to other satellite internet providers. The satellites being in Low Earth Orbit, while allowing such positive results require many satellites at a relatively low orbit, to other satellites. These lower satellites tend to have less of a shelf life than their higher counterparts. This shorter shelf life is due, mostly, to the upper atmosphere's drag on these satellites. Models that predict this decay in satellite's orbit tend to be off by 10\%. A dynamic RL system might be able to respond to variations in the orbital decay. Besides, an RL system may be less computationally expensive, and in fact, research has gone into this area, which I'll present in section 1.1.

Reinf

\section{Related Work}
