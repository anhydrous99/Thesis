\chapter{Introduction}\label{ch:intro}

Reinforcement Learning is starting to be prevalent in many fields and applications, including Autonomous Driving, Robotics, Transportation, in Science and Engineering, and many more. A lot of recent research has started in applying Reinforcement Learning in Space and space fairing vehicles. The benefits are better computational requirements from switching from classical numerical methods to a Reinforcement Learning based approach. The goal is to create an environment for detailed evaluations of RL algorithms, current, and future, in Station-Keeping, maintaining a satellite's orbit.

With SpaceX's Starlink, their satellite internet project, we are putting an increasing number of satellites in Low Earth Orbit. In the case of Starlink, this allows their satellites to deliver low latency high bandwidth internet compared to other satellite internet providers. The satellites being in Low Earth Orbit, while allowing such positive results require many satellites at a relatively low orbit, to other satellites. These lower satellites tend to have less of a shelf life than their higher counterparts. This shorter shelf life is due, mostly, to the upper atmosphere's drag on these satellites. Models that predict this decay in satellite's orbit tend to be off by 10\%. A dynamic RL system might be able to respond to variations in the orbital decay. Besides, an RL system may be less computationally expensive, and in fact, research has gone into this area, which I'll present in section~\ref{sec:relatedwork}.

The creation of standard and freely available environments and algorithm benchmarking for the particular use case, here orbital station keeping, can greatly acceleration RL research in said area and allows for easier RL algorithm comparison and creation of tailored RL algorithms specifically for the domain. Having a preset, ready made, environment for Orbital Station Keeping allows future researchers to more quickly assess RL algorithms. This is what this thesis aims to create.

\section{Related Work}\label{sec:relatedwork}

There is a lot of research being done in utilizing Reinforcement Learning in various aspects of satellite and spacecraft navigation, replacing the classical and computationally expensive navigation algorithms. An example is \cite{oestreich2020autonomous} where Reinforcement Learning is used to perform docking maneuvers and \cite{broida2019spacecraft} where it is used provide guidance in spacecraft rendezvous. Reinforcement Learning has also been used to provide guidance in orbital transfer, \cite{miller2019low}.

Current research in this area is really being progressed because of the Artemis Project, which is tasked with landing astronauts on the moon again. The mission includes the creation of a small space station around lunar orbit. The L-1 Lyapunov Orbit \cite{rubinsztejn_2019}, which is the favorite orbit, is an orbit around the Lagrange Point between the Moon and Earth. Orbital Transfers between an earth orbit and the L1-Lyapunov Orbit is therefore a focus for researchers \cite{sullivan2020using, lafarge2020guidance, lafarge2020autonomous}. 
