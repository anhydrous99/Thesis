\documentclass{article}

\title{The Restricted Three-Body Problem}
\author{Armando Herrera}

\begin{document}
\maketitle

Here I strive to describe the mechanics of the three-body problem. In order to go over the three-body problem, we must first go over the underlying physics, describe the simpler single body problem, two body problems, the three body problem, and finally the restricted three-body problems.

\section{Classical Mechanics}

With classical mechanics it is possible to describe the motion of most object in the universe. Using it, we can describe the motions of certain object in non-extreme environments. For instance, we can use it to describe the motion of a bicycle, the trajectories of billiard balls, etc... . It's limitation start at speeds approaching light speed, in the quantum realm, or near super-massive objects like a black hole. It is built on top of certain physical concepts, some formulated by Issac Newton.

Objects have certain conserved properties to them, under Classical Mechanics, a constant mass, a relative position in space, and a relative velocity in space. We can say that an objects velocity is the rate of change of the position of said object. So, in terms of a derivative: $$\vec{v}=\dot{\vec{r}}$$ 

Here I denote $\vec{r}$ to be the position of the object and $\vec{v}$ to be the velocity of the object. The $\vec{}$ denotes the fact that they are vectors in euclidean space and $\dot{}$ is equivalent to saying $\frac{d}{dt}$.

Similarly, the acceleration of an object can be defined as the rate of change of velocity of said object or the rate of change of the rate of change of the object's position. $$\vec{a}=\dot{\vec{v}}=\ddot{\vec{r}}$$ 
Here I use $a$ to the acceleration and $\ddot{}$ as the second derivative, $\frac{d^2}{dt^2}$.

Newton's second law of motion describe the relationship between force and acceleration, showing that the net force $F$ applied to an object is equal to the mass $m$ multiplied by the acceleration. $$\vec{F}=m\vec{a}$$

Now, there are categories of forces, Contact Forces and Action-at-a-Distance Forces. The differentiating factor between the two categories is that the Contact Forces, as the name implies, requires two object's contact and Action-at-a-Distance Forces are applied, again as the name implies, at a distance. In the vacuum of space the main force active is the Force of Gravity.

The force of gravity, and most Action-at-a-Distance Forces, exist by the inverse-square law, where it's amount is inversely proportional to the distance from the source of the force, here gravity. $$F\propto \frac{1}{r^2}$$ In 1687 Newton in Newton's Principia postulated what is now called Newton's Law of Universal Gravitation \cite{rohrlich_1999}. $$F=G\frac{m_1m_2}{r^2}$$

Newton's Law of Universal Gravitation gives us the key to the story between objects in the vacuum of space. 

\end{document}